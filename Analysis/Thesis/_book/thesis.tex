% This is the Oregon State University LaTeX template. To the best of my
% knowledge, most of the work was done by those acknowledged in beavtex.cls.

%%
%% Preamble
%%
% \documentclass{<something>} must begin each LaTeX document
\documentclass[double,12pt]{beavtex}
% Added by CII
\usepackage{graphicx,latexsym, array, wrapfig}
\usepackage{amsmath}
\usepackage{amssymb,amsthm}
\usepackage{longtable,booktabs,setspace}
\usepackage[hyphens]{url}
\usepackage[colorlinks = true, 
			urlcolor = blue,
			linkcolor = black,
			citecolor = black,
			anchorcolor = black]{hyperref}
\usepackage{lmodern}
\usepackage{float}
\floatplacement{figure}{H}
% End of CII addition
\usepackage{rotating} % Package added to allow the rotation of figures and chart
                      % on a page, {sidewaysfigure} command
\usepackage{tablefootnote} % Packaged added to allow footnotes in the tabular
                           % environment, use \tablefootnote command

% This has to do with a default pandoc thing
% http://tex.stackexchange.com/a/258486/77699
\providecommand{\tightlist}{%
  \setlength{\itemsep}{0pt}\setlength{\parskip}{0pt}}

% Added by CII (Thanks, Hadley!)
% Use ref for internal links
\renewcommand{\hyperref}[2][???]{\autoref{#1}}

\def\sectionautorefname{Section}
\def\subsectionautorefname{Subsection}
% End of CII addition

% Added by CII
\usepackage{caption}
\captionsetup{width=5in}
% End of CII addition

\title{Response of Stream Macroinvertebrate Community to Canopy-opening
Manipulations} % {An Analysis of Something}
\author{Cedar Mackaness} % {Joseph A. Student}
\degree{Honors Baccalaureate of Science} % {Master of Science}
\doctype{Thesis}
\submitdate{January 1, 2013} % {January 1, 2013}
\commencementyear{2020} % {2013}

\department{Environmental Science} % {Nuclear Engineering and Radiation Health Physics}

\depttype{College of Earth, Ocean and Atmospheric Sciences} % {School}

\depthead{Dave Lytle} % {Director}

\major{Environmental Science} % {Radiation Health Physics}

\advisor{Dana Warren} % {Jane R. Professor}

\abstract{Stream light availability is an important factor influencing aquatic
food webs. In forested headwaters, stream algal production is highly
light-limited, and an increase in light often enhances benthic algal
growth, which in turn increases food availability for primary consumers
in the stream. In forested headwater streams, light availability is
almost entirely mediated by the canopy structure of stream-side
vegetation. Over the last century, many streamside forests in the
Pacific Northwest were heavily harvested, leaving dense regenerating
stands for the time being. Under current conditions, the dense closed
canopies, allow for limited primary production, and a low abundance of
invertebrates that feed on stream algae. We investigated the response of
benthic periphyton, stream macroinvertebrates, and prey consumption by
trout to a release from light limitation in a paired-reach study design.
We hypothesized that increased light availability promotes elevated
algal production which causes the invertebrate community to shift toward
scraper dominance, and predicted that this change in community structure
is detectable in the diets of trout. We found that the presence of a
canopy gap had little influence on the invertebrate community, and this
lack of change was not being masked by increased consumption of grazing
invertebrates in summer trout diets.}
\acknowledgements{I would like to thank Allison Swartz, Dana Warren and the whole Warren
lab for several years of sometimes-grueling-always-fun work, and the
opportunity and support to work on this thesis project. Additionally, I
would like to acknowledge both Dave Roon and Dave Lytle, my thesis
committee members, for their novel insights and unique perspective. The
parts of this thesis that I am most proud of came about through
collaboration with all of the people mentioned above, and of course it
was all made possible through the love and encouragement of my family.}








\begin{document}

\maketitle
\mainmatter


  \chapter*{Introduction}\label{introduction}
  \addcontentsline{toc}{chapter}{Introduction}
  
  In forested systems, streams and their biota are intrinsically linked to
  riparian vegetation (Vannote, Minshall, Cummins, Sedell, \& Cushing
  (\protect\hyperlink{ref-Vannote1980}{1980})). Stream food webs depend on
  direct carbon subsidies from the terrestiral environment in the form of
  both leaf litter and terrestrial invertebrates (Wipfli
  (\protect\hyperlink{ref-Wipfli1997}{1997})), but riparian controls on
  stream systems aren't limited to organic inputs. Riparian canopy cover
  also has an indirect effect on stream food webs through the control of
  light available for benthic primary production. In the Pacific Northwest
  (PNW) region of North America, riparian forests have changed
  substantially in the past half century. In response to a legacy of heavy
  harvesting (Pan et al. (\protect\hyperlink{ref-Pan2011}{2011})),
  riparian forest protections have created dense second-growth vegetation
  along streams in contrast with structurally complex old-growth forests
  containing multiple canopy gaps (D. R. Warren et al.
  (\protect\hyperlink{ref-Warren2016}{2016})). The dense vegetation in
  these regenerating forests decreases light availability and limits
  benthic primary production (M. J. Kaylor, Warren, \& Kiffney
  (\protect\hyperlink{ref-Kaylor2017FS}{2017})). As forest stand
  development continues natural disturbances and individual tree
  mortalities will increase canopy heterogeneity through the introduction
  of gaps. To understand how aquatic food webs respond to an increase in
  light associated with canopy gaps, we investigate the response of
  macroinvertebrates and fish feeding to canopy-opening manipulations.
  
  Light, and its impact on primary productivity in streams is of
  particular interest because autochthonous carbon can be
  disproportionately represented in consumer biomass relative to its
  availability in aquatic systems (Lau, Leung, \& Dudgeon
  (\protect\hyperlink{ref-Lau2009}{2009}), McCutchan \& Lewis
  (\protect\hyperlink{ref-McCutchan2002}{2002})). In forested headwaters
  specifically, basal carbon availability is dominated by leaf litter
  (McCutchan \& Lewis (\protect\hyperlink{ref-McCutchan2002}{2002}));
  however, energetically, algae is a higher quality food source and is
  preferentially assimilated into higher trophic levels (Macarelli \&
  others (\protect\hyperlink{ref-Macarelli2011}{2011})). Primary consumers
  mediate basal carbon availability for higher trophic levels, and in
  streams, primary consumers are dominated by macroinvertebrates, a major
  food resource for insectivorous fish. Because macroinvertebrates play a
  crucial role in mediating food web interactions, understanding their
  community dynamics and functional diveristy can provide key insights
  into broader ecosystem functioning. Invertebrates in the scraper
  functional feeding group in particular have evolved specialized
  mouthparts for consuming benthic algal biofilms (periphyton), and
  increases in algal production in high light areas can elicit a positive
  response among these scraping taxa (Liess, Le Gros, Wagenhoff, Townsend,
  \& Matthaei (\protect\hyperlink{ref-liess2012}{2012})).
  
  Macroinvertebrate community data have historically been used to evaluate
  stream health. Indicies such as the B-IBI (benthic index of biological
  integrity) rely on total taxa richness and taxa richness of key
  families, such as Plecoptera, Ephemeroptera and Trichoptera, to evaluate
  the biological condition of streams. More broadly, an assesement of the
  whole community can be used to evaluate overall food web and ecosystem
  responses to a multitude of variables. For example, studies using
  nonmetric multidimensional scaling (NMS) has been used to assess
  community responses along a variety of environmental gradients (M. B.
  Cole, Russell, \& Mabee (\protect\hyperlink{ref-Cole2003}{2003}),
  Purcell et al. (\protect\hyperlink{ref-Purcell2009}{2009})).
  
  In headwater streams, the benthic invertebrate community represents the
  primary food source for trout, although this may vary seasonally
  (Syrjänen, Korsu, Louhi, Paavola, \& Muotka
  (\protect\hyperlink{ref-Syrjanen2011}{2011})). Interspecific
  interactions between trout and invertebrates can alter the benthic
  community and cascade to lower trophic levels (Peckarsky \& McIntosh
  (\protect\hyperlink{ref-Peckarsky1998}{1998})). In headwater streams
  trout are oportunistic foragers, eating whatever is available in their
  habitat. Cutthroat trout, the dominant fish species in Cascade
  headwaters, typically feed from the water column using visual cues to
  capture prey. Because salmonids are visual predators, their feeding
  efficiency can be influenced by light conditions and visibility
  (Wilzbach, Cummins, \& Hall
  (\protect\hyperlink{ref-Wilzbach1986}{1986})), therefore gaps have the
  potential to affect fish feeding not only though potential increases in
  scraper invertebrate food resources, but also by increasing foraging
  capture rates of all taxa and functional feeding groups.
  
  Clear-cut harvests without buffers and the resultant reach-level
  increase in stream light can increase stream primary and secondary
  productivity, but increases in light also lead to increases in
  temperature, and cutting to the stream edge can increase sediment loads.
  Given these negative impacts, clear-cutting adjacent to streams is no
  longer a common practice in the Pacific Northwest. In unmanaged forests,
  and in recently implemented riparian buffers, stands are recovering from
  past land clearing, and are currently in the early to mid-seral stage of
  development with dense homogenous canopy cover and low stream light (M.
  J. Kaylor et al. (\protect\hyperlink{ref-Kaylor2017FS}{2017})). Canopy
  gaps will begin developing naturally along streams as stands mature, and
  restoration efforts focused on emulating natural disturbance may
  expedite forest shifts toward late-succession and old-growth structural
  conditions (Kreutzweiser, Sibley, Richardson, \& Gordon
  (\protect\hyperlink{ref-Kreutzweiser2012}{2012})). While studies on
  reach-scale forest clearing demonstrate a clear response in benthic
  primary producers, invertebrates, and trout to release from light
  limitation (Murphy \& Hall (\protect\hyperlink{ref-Murphy1981}{1981})),
  this does not reflect future riparian conditions in most forested
  landscapes. Rather, as stands progress toward late succesional forest
  structure, localized light patches (as opposed to large openings) will
  become increasingly prevelant. The more moderate influence of these
  small canopy gaps on stream ecosystems has not been widely investigated,
  especially in an experimental context.
  
  We implemented a two-year long before-after, control-impact study
  designed to detect and accurately capture the effect of canopy gaps on
  aquatic ecosystems. In this work, we hypothesized that primary
  production would increase when canopy gaps were created, and this would
  cause the invertebrate community to shift toward more scraping taxa.
  However, we expect the response to be dampened in comparison to observed
  responses in large scale riparian clearing studies. In addition to
  evaluating the macroinvertebrate community, we quantified trout diets
  with the expectation that shifts in the invertebrate community would be
  reflected proportionally in the diet of opportunistic foraging of trout,
  but also to ensure that a potential signal in invertebrate scraper taxa
  was not being masked by increased foraging of apex consumers.
  
  \chapter*{Methods}\label{methods}
  \addcontentsline{toc}{chapter}{Methods}
  
  \section*{Study location}\label{study-location}
  \addcontentsline{toc}{section}{Study location}
  
  The study consisted of five reach pairs on five replicate streams in the
  western Cascade Mountains of Oregon. Each reach pair consisted of one
  treatment reach and one reference reach. Two of the reach pairs (W-100,
  W-113) are located on private Weyerhaeuser Co. land, and three (LOON,
  CHUCK, MCTE) are located on U.S. Forest Service land, one of which
  (MCTE) is situated in the HJ Andrews Experimental Forest. Stream reaches
  were 90 meters in length and treatment gaps were 20 to 40 meters in
  diameter and situated approximately around meter thirty of treatment
  reaches. Sites had a buffer between stream reach pairs to limit any
  effects of the upstream reach on downstream conditions.
  
  All of the streams are wadeable, fish-bearing streams with bankfull
  widths of 1-6 meters. Fish-bearing streams were purposefully selected to
  provide management-relevant results for key species such as salmonids.
  Additionally, streams of this size comprise roughly 70\% of total stream
  length in forested catchments. The streams run through 40-60-year-old
  riparian forests regenerating from previous harvest. These forests have
  a homogenous canopy structure with heavy understory shading, as defined
  by their early to mid seral stage. Small streams also provide ease of
  sampling and maximize the effect of a canopy opening manipulation since
  small streams may be completely shaded by overhead vegetation due to
  their high edge to area ratio.
  
  \begin{table}[t]
  
  \caption{\label{tab:table1}Study site attributes}
  \centering
  \resizebox{\linewidth}{!}{
  \begin{tabular}{lrrrrr}
  \toprule
  Stream & Elevation (m) & Bankful Width (m) & Base flow (L s-1) & Latitude & Longitude\\
  \midrule
  CHUCK & 833.0 & 5.19825 & 21.0 & 43.95362 & -122.1136\\
  LOON & 721.2 & 4.13170 & 12.5 & 43.95362 & -122.1833\\
  MCTE & 867.0 & 2.20000 & 5.0 & 44.25454 & -122.1667\\
  W-100 & 441.0 & 5.39000 & 43.9 & 44.19813 & -122.4930\\
  W-113 & 537.0 & 3.30000 & 9.1 & 44.19289 & -122.5107\\
  \bottomrule
  \end{tabular}}
  \end{table}
  
  \section*{Study Design}\label{study-design}
  \addcontentsline{toc}{section}{Study Design}
  
  The before-after, control-impact (BACI) study design lends itself to
  experimental field studies by accounting for natural variations between
  sites. By taking the difference of a given variable between the paired
  reaches and comparing the change in the difference from pre to
  post-treatment years, we account for both spatial and temporal
  variation. For the BACI analyses, a sample unit refers to a whole stream
  including both treatment and reference reaches because the metric of
  interest for BACI is the difference between the two reaches. Therefore,
  we have five sample units with two repeated measures, pre and
  post-treatment. To test for effects of the gap treatment, we quantify
  and assess changes in the reach differences between the two years.
  Samples were collected during summer 2017 and summer 2018 with
  pre-treatment data gathered during summer 2017 and post-treatment data
  gathered during summer 2018. Canopy gaps were cut in the treatment reach
  during the winter of 2017-18 to permit adequate time for response to the
  canopy manipulation at all sites besides MCTE. At MCTE gaps were cut at
  the end of summer 2017 after data collection.
  
  \section*{Data Collection}\label{data-collection}
  \addcontentsline{toc}{section}{Data Collection}
  
  \subsection*{Light}\label{light}
  \addcontentsline{toc}{subsection}{Light}
  
  Daily, photosynthetically active radiation (PAR) was estimated from
  flourescein decay rate over a twenty-four hour period following methods
  in Warren et al. Flourescein dyes were prepared by diluting to 400 g
  L\textsuperscript{-1} with DI water and buffering with 40 g
  L\textsuperscript{-1} of aquarium salt. Once the dye was prepared, we
  filled 3.7mL glass vials and stored them in the dark until deployment.
  At each study reach hree replicate vials were deployed every five
  meters, and retrieved twenty-four hours later. Because flourescence of
  fluorescein changes with temperature (Bechtold, Rosi-Marshall, Warren,
  \& Cole (\protect\hyperlink{ref-Bechtold2012}{2012})), vials were left
  in the dark until they reached room temperature. Flouresence was then
  measured using a flourometer (Turner Designs, San Jose, California), and
  the twenty-four hour decay rate was converted to daily
  photosynthetically active radiation (PAR) using the relationship in (D.
  R. Warren, Collins, Purvis, Kaylor, \& Bechtold
  (\protect\hyperlink{ref-Warren2017}{2017})).
  
  \subsection*{\texorpdfstring{Chlorophyll
  \emph{a}}{Chlorophyll a}}\label{chlorophyll-a}
  \addcontentsline{toc}{subsection}{Chlorophyll \emph{a}}
  
  In each study reach, three ceramic tiles (15 cm x 15cm) were placed
  every 10 meters and left for 4 weeks before they were collected to allow
  periphyton communities to establish. Tiles were placed in riffle
  sections at a depth of 10-25 cm to keep them from silting over. All
  tiles were deployed in mid-July, and the tiles were deployed at the
  control and treatment reaches of each stream at the same time to keep
  within unit measures consistent. After collection, tiles were kept in
  the dark, submerged in water for two hours to avoid potential
  photosasturation issues with the \emph{in situ} chlorophyll \emph{a}
  measurements Chlorophyll \emph{a} (abbreviated as Chla for the remainder
  of this text) concentrations were then quantified using a
  BenthoTorch\textsuperscript{TM} (BBE Moldaenke GmbH), a portable field
  instrument used for the quantification of chlorophyll \emph{a}
  fluorescence on different substrates.
  
  \subsection*{Benthic Invertebrate
  Sampling}\label{benthic-invertebrate-sampling}
  \addcontentsline{toc}{subsection}{Benthic Invertebrate Sampling}
  
  Three benthic invertebrate samples were taken at each stream reach at
  meters 15, 45, 75, or the closest area with non-boulder substrate.
  Samples were collected once per year over the course of one week using a
  Surber sampler with a .09 m\textsuperscript{2} sampling area. Substrate
  was disturbed to a depth of approximately four inches for one minute.
  The sample was then preserved in 95\% ethanol for identification and
  enumeration in the lab.
  
  In the lab, the three benthic samples per reach were combined into a
  single pooled sample for each reach. The pooled sample was then
  subsampled using a Caton tray. Squares \(\frac{1} {30}\) the area of the
  Caton tray were randomly sampled until the cutoff of 300 individuals or
  greater was reached. Benthic invertebrates were then identified down to
  genus or the lowest taxonomic unit (LTU) for cryptic taxa such as
  Chironomidae primarily using Merritt, Cummins, \& Berg
  (\protect\hyperlink{ref-Merritt2008}{2008}). Counts from subsamples were
  then converted to densities using the following formula:
  
  \begin{equation}
  \frac{1}{3*s*0.09}
  \end{equation}
  
  where \(s\) is the fraction subsampled, 0.09 is the area of the Surber
  sampler in square meters, and the result is divided by three because
  three samples from meters fifteen, forty-five and seventy-five were
  pooled.
  
  For NMS and other community analyses, singleton taxa (taxa occurring in
  only one reach) were removed from the original matrix and density values
  were log transformed to reduce the effect of abundant taxa
  (Chironomidae, \emph{Baetis}, \emph{Micrasema}) on community
  relationships by applying the formula \(ln(n + 1)\) where \(n\). The
  resulting matrix of benthic invertebrates at the LTU level of
  identification (20 reaches by 64 taxa) was then used for analysis.
  Functional feeding groups were assigned using the trophic relationships
  of each taxon as identified in Merritt et al.
  (\protect\hyperlink{ref-Merritt2008}{2008}), and raw density values were
  used for FFG analyses because sparse densities were not a concern with
  aggregate functional groups.
  
  During Chla tile collection at the two streams with snails as the
  dominant scraper, the number of snails (Juga) and cased caddisfly
  (observed taxa being Uenoidae and Glossosomatidae primarily) on each
  tile were recorded and then removed. before taking readings with a
  BenthoTorch\textsuperscript{TM}.
  
  \subsection*{Trout Diets}\label{trout-diets}
  \addcontentsline{toc}{subsection}{Trout Diets}
  
  Trout diets were collected during the post-treatment year. Trout diets
  were collected during three-pass depletion of fish population estimates
  and were only taken from a subset of fish greater than 100 mm in length.
  Fish were anesthetized using AQUI-S and gastric lavaged. Stomach
  contents were evacuated by inecting water into the fish stomach using a
  piece of small plastic tubing attached to a syringe. Diet samples were
  collected in filter paper and preserved in 95\% ethanol for lab
  processing.
  
  All trout diets were processed (9 to 13 diets per reach) with aquatic
  invertebrates identified down to the family level and terrestrial
  invertebrates identified to order. Because the number of fish dieted in
  each reach varied, the average of all fish diets was used. The resulting
  matrix was then filtered for aquatic species and appended to a matrix of
  2018 benthic invertebrate families (10 reaches by 38 families),
  producing a matrix of 20 sample units (SU's) by 40 families consisting
  of both fish diets and benthic samples. Singleton taxa were then removed
  to create a matrix of combined diet and benthic families of 20 SU's by
  36 families. At this point, the combined matrix was relativized by row
  maxima to compensate for the difference between benthic
  sampling---measured in density per m2---and fish diets.
  
  \section*{Data Analysis}\label{data-analysis}
  \addcontentsline{toc}{section}{Data Analysis}
  
  \subsection*{BACI Analysis}\label{baci-analysis}
  \addcontentsline{toc}{subsection}{BACI Analysis}
  
  The BACI analysis was performed in R (R Core Team
  (\protect\hyperlink{ref-R-base}{2018})), and consisted of calculating
  reach-pair differences by subtracting the control reach value from the
  treatment reach value. Reach differences were calculated for light,
  chla, total invertebrate density and invertebrate densities by
  functional feeding group. A paired t-test with 4 degrees of freedom was
  then performed for each metric by subtracting the reach difference from
  the pre-treatment year from the difference value in the post-treatment
  year for each stream assuming the difference between the two reach
  ratios should be zero.
  
  \subsection*{Community Analysis}\label{community-analysis}
  \addcontentsline{toc}{subsection}{Community Analysis}
  
  Community analyses were performed in PC-ORD (McCune \& Mefford
  (\protect\hyperlink{ref-PC-ORD}{2016})) and R (R Core Team
  (\protect\hyperlink{ref-R-base}{2018})) using the Vegan package (Oksanen
  et al. (\protect\hyperlink{ref-vegan}{2018})). Blocked multi-response
  permutation procedure (MRBP) was used to assess differences between
  treatment and control reaches in the pre and post treatment years. MRBP
  was followed up with blocked indicator species analysis (ISA) to
  determine underlying taxa driving any grouping detected by MRBP. The
  combined benthic and diet community matrix was subsequently tested for
  any differences between treatment and control reaches and benthic versus
  diet taxa representation using the same MRBP and ISA methods.
  
  To test for any pre-treatment reach differences in 2017, MRBP was run on
  2017 data only with Treatment as the two a priori groups and blocked by
  Stream. The 2018 post-treatment data was then assessed using the same
  MRBP grouping and blocking. MRBP is a nonparametric method used to test
  for differences between groups. This method accommodates paired or
  blocked study designs by accounting for variation related to study
  design variables that have little bearing on the question being
  addressed. In this case, MRBP accounts for any between-stream variation.
  MRBP outputs a p-value for the observed within-group distance (smaller
  distances constituting stronger grouping) by shuffling SU's between
  groups to generate a distribution of possible within-group distances
  (McCune, Grace, \& Urban (\protect\hyperlink{ref-McCune2002}{2002})).
  
  The follow-up ISA calculates an indicator value (IV) for each species.
  The IV is a composite of a taxon's fidelity and exclusivity to a group.
  If a taxon is consistently abundant in one group and never present in
  any other, then it would receive a high IV. Conversely, a taxon rarely
  abundant in SU's of one group and present in other groups would receive
  a low IV (McCune et al. (\protect\hyperlink{ref-McCune2002}{2002})). A
  Monte Carlo test of 1,000 permutations of the taxa matrix was used to
  generate a p-value for each taxon's IV.
  
  Nonmetric multidimensional scaling (Kruskal
  (\protect\hyperlink{ref-Kruskal1964}{1964})) was used to visually assess
  differences between the treatment and control reach communities, and
  quantify the relationship between the synthetic community variable and
  Chla . Sorensen distance was used for both ordinations to reduce the
  impact of outliers. The ordination was rotated to maximize the
  environmental variable Chla along axis 1. A random start was used and
  the real data were run 250 times to ensure an absolute stress minima was
  reached. A Monte Carlo test with 100 permutations was used to generate a
  p-value for the probability that the final ordination has a lower than
  expected stress value.
  
  \subsection*{Analysis of Trout Diets}\label{analysis-of-trout-diets}
  \addcontentsline{toc}{subsection}{Analysis of Trout Diets}
  
  Trout diets were collected in the post-treatment year, which limits
  analysis to a comparison of reference and treatment reaches without the
  BACI control on inherent reach differences. We performed paired t-tests
  for the abundance of each functional feeding group represented in the
  diets of trout in the reference and the treatment reach, and on the
  modified Ivlev's selectivity index (as defined in Jacobs
  (\protect\hyperlink{ref-Jacobs1974}{1974})) for each FFG.
  
  \chapter*{Results}\label{results}
  \addcontentsline{toc}{chapter}{Results}
  
  \section*{Light and Chlorophyll}\label{light-and-chlorophyll}
  \addcontentsline{toc}{section}{Light and Chlorophyll}
  
  In 2017, before treatment, the average daily PAR reaching the stream
  benthos among the five streams was consistently low in both reference
  and treatment reaches and there was an average difference between the
  treatment and reference reach of -0.16 mol m\textsuperscript{-2}
  day\textsuperscript{-1}. In 2018, after gaps were cut, light went up by
  2.60 mol m\textsuperscript{-2} day\textsuperscript{-1} on average in the
  treatment reach compared to the reference reach (Figure 1) resulting in
  a final yearly difference between reach differences of 2.77 mol
  m\textsuperscript{-2} day\textsuperscript{-1} (p-value = 0.019, t-value
  = -3.83).
  
  Again, for chla, values across all sites in the pre-treatment year were
  low (mean = 0.095 ug cm\textsuperscript{-2}), and there was little
  difference between reaches. After gaps were cut in the post-treatment
  year, Chla values went up by 0.44 ug cm\textsuperscript{-2} in the gap
  reach, and only 0.175 ug cm\textsuperscript{-2} in the reference reach
  (final BACI difference = 0.265 ug cm\textsuperscript{-2}, p-value =
  0.002).
  
  \begin{figure}
  
  {\centering \includegraphics[width=0.6\linewidth]{Figures/Vars_Reach_Diffs} 
  
  }
  
  \caption[Light and Chla reach differences in the pre and post-treament years \label{Chla-Light}]{Light and Chla reach differences in the pre and post-treament years \label{Chla-Light}}\label{fig:unnamed-chunk-1}
  \end{figure}
  
  \section*{\texorpdfstring{\emph{Juga} on
  Tiles}{Juga on Tiles}}\label{juga-on-tiles}
  \addcontentsline{toc}{section}{\emph{Juga} on Tiles}
  
  In the pre-treatment year, the average density of \emph{Juga} on tiles
  among the two streams with \emph{Juga} present was 24.44 snails per
  m\textsuperscript{2} with little difference between the control and
  treatment reaches. In the post treatment year the average snail density
  in the treatment reach increased by 204.44 snails per
  m\textsuperscript{2}, whereas snail density in the control reach only
  increased by 88.89 snails per m\textsuperscript{2}. Snail abundance at
  these two streams was moderately associated with Chla
  (r\textsuperscript{2} = 0.3204, p = 0.00547), but saw the largest BACI
  response in meters ten and twenty, slightly upstream of the gap
  treatment.
  
  \section*{Benthic Invertebrate
  Community}\label{benthic-invertebrate-community}
  \addcontentsline{toc}{section}{Benthic Invertebrate Community}
  
  There was little difference between benthic invertebrate communities in
  the treatment and reference reaches in the pre-treatment year (MRBP: A =
  0.041, p = 0.071), or the post-treatment year (A = -0.022, p = 0.838).
  The results from the NMS ordinations support the results of the MRBP
  (Figure 2).
  
  \begin{figure}
  
  {\centering \includegraphics[width=0.6\linewidth]{Figures/NMS-Benthic-1} 
  
  }
  
  \caption[NMS of each reach in invertebrate community space]{NMS of each reach in invertebrate community space. Each point represents a single stream reach, shapes identify stream, and color identifies treament and year.}\label{fig:unnamed-chunk-2}
  \end{figure}
  
  The NMS ordination of benthic invertebrates converged on a 2D solution
  with a final stress of 12.031. BenthoTotal (total chlorophyll values
  from the BenthoTorch) and YearTreatQ (a binary variable coded with 1's
  for 2018 treated reaches and 0's for all other reaches) both had
  positive r\textsuperscript{2} values with axis 1 (YearTreatQ
  r\textsuperscript{2} = 0.272, BenthoTotal r\textsuperscript{2} = 0.304).
  
  \textbf{Add relationship of individual taxa to axes and ISA analysis.}
  
  \section*{Invertebrate Functional Feeding
  Groups}\label{invertebrate-functional-feeding-groups}
  \addcontentsline{toc}{section}{Invertebrate Functional Feeding Groups}
  
  Collector gatherers were by far the most abundant functional feeding
  group in the post-treatment year for both reaches at all sites. This
  does not appear to be due to the treatment of the gaps since we see
  heightened collector gatherer response in the reference reach as well.
  Collector filterers were typically the least abundant FFG in any stream
  or year. No FFG had a significant response across all streams. Scraping
  invertebrates only showed a positive response to the gap in MCTE with
  all other streams having a moderately negative BACI response. When we
  treat streams as independent replicates and perform a t-test of total
  invertebrate density response and the density response of each FGG
  individually, we find that only collector filterers had a statistically
  significant response (Table 2).
  
  \begin{figure}
  
  {\centering \includegraphics[width=0.6\linewidth]{Figures/AvgFFGratio-1} 
  
  }
  
  \caption[Log ratio of treatment reach divided by control reach for each FFG]{Log ratio of treatment reach divided by control reach for each FFG. CF = Collector-filterer, CG = Collector-gatherer, SH = Shredder, SC = Scrapers, SCe = Edible Scrapers, SCi = Inedible Scrapers, P = Predators, and All Bugs is the total macroinvertebrate density}\label{fig:unnamed-chunk-3}
  \end{figure}
  
  \begin{table}[t]
  
  \caption{\label{tab:ffgtable}BACI t-test results for various metrics}
  \centering
  \begin{tabular}{lrr}
  \toprule
  FFG & t-value & p-value\\
  \midrule
  SH & 0.0711617 & 0.9450224\\
  P & 0.5161519 & 0.6224931\\
  SCe & -1.5465440 & 0.1624379\\
  CG & 0.5978468 & 0.5666133\\
  SCi & 0.8552979 & 0.4216123\\
  \addlinespace
  CF & 2.1321982 & 0.0656781\\
  All Bugs & 0.8400000 & 0.4300000\\
  \bottomrule
  \end{tabular}
  \end{table}
  
  \section*{Trout Diet}\label{trout-diet}
  \addcontentsline{toc}{section}{Trout Diet}
  
  \begin{center}\includegraphics[width=0.6\linewidth]{Figures/Diet-FFG-D-1} \end{center}
  
  \begin{center}\includegraphics[width=0.6\linewidth]{Figures/Diet-Fam-D-1} \end{center}
  
  \chapter*{Discussion}\label{discussion}
  \addcontentsline{toc}{chapter}{Discussion}
  
  \section*{large idea to keep in mind}\label{large-idea-to-keep-in-mind}
  \addcontentsline{toc}{section}{large idea to keep in mind}
  
  Gaps are, by definition, open canopy patches in a larger forested
  system. While localized responses beneath a gap may occur, we were
  particularly interested in whether the effect of an individual canopy
  gap could be measured at the stream reach scale. Studies have found that
  systems with multiple gaps (Wootton
  (\protect\hyperlink{ref-Wootton2012}{2012})), or with patches of high
  shade (Heaston, Kaylor, \& Warren
  (\protect\hyperlink{ref-Heaston2018}{2018})), had an effect on the
  overall invertebrate community. Yet, significant localized responses to
  increases in light beneath a single gap may not translate to significant
  system-wide responses at the stream or even the reach level. Our study
  design emphasizes the effects of gaps that comprise only a fraction of a
  stream reach, focusing on the integrated effect of small gaps embedded
  in a larger forested environment.
  
  \section*{General overview}\label{general-overview}
  \addcontentsline{toc}{section}{General overview}
  
  While light and Chla responded as expected to an opening of the riparian
  canopy, our reach-scale metrics for the invertebrate community do not
  show a response to increases in primary production. This does not match
  our original hypothesis that increases in high quality algae would lead
  to increased abundance of scraping invertebrate taxa. Instead, we found
  the magnitude of the invertebrate response was not large enough to
  manifest at the reach scale and that fish consumption was most likely
  not masking changes in community compisiton or functional feeding group
  relative abundance. In the one taxa group for which we did evaluate
  local responses, snails, we saw a strong response to the treatment in
  meters adjacent to the gap treatment, but our reach-scale metrics for
  other invertebrate community members showed little BACI response.
  
  \section*{light and Chla}\label{light-and-chla}
  \addcontentsline{toc}{section}{light and Chla}
  
  While light increases in our study were large due to previous heavy
  shading, they were not outside the realm of what occurs naturally in
  these systems. While these small-scale distrubances may have localized
  impacts on stream biota, the reach scale response seems to be limited.
  Previous studies on forest shading, indicate a strong linear
  relationship between light and GPP, yet our increases in light show
  limited response. This may indicate potential photosaturation limiting
  algal response to gaps.
  
  \section*{Invertebrates FFG response}\label{invertebrates-ffg-response}
  \addcontentsline{toc}{section}{Invertebrates FFG response}
  
  The reach-scale metrics seem to be in contradiction with previous
  studies on stream light (Heaston et al.
  (\protect\hyperlink{ref-Heaston2018}{2018}), Kaylor \& Warren
  (\protect\hyperlink{ref-Kaylor2017Eco}{2017}), Wootton
  (\protect\hyperlink{ref-Wootton2012}{2012})), but these studies focused
  on the immediate, within-treatment response of invertebrates and fish to
  various alterations to light availability. In that regard, our snails
  responded as expected, but the relative size of our canopy
  manipulations, one similar to small scale natural disturbances and
  individual tree mortality, limited reach-level trophic responses.
  
  \section*{Community, why we feel justified using
  FFG's}\label{community-why-we-feel-justified-using-ffgs}
  \addcontentsline{toc}{section}{Community, why we feel justified using
  FFG's}
  
  Community PCA results what have others (in Dave's lab) found and what
  did we find here? How might these results change if you were to look at
  other seasons? Overall communities were more similar within a stream
  than across treatments. This is to be expected but we thought tehre
  would be a shift in the same direction that would show up. We did see
  trends in SC that fit our hypothesis, that were not significant, however
  what is interesting is that with such different communities between
  streams the shifts that we saw were a result of changes across a wide
  range of taxa, not just one taxa group changing. So even though there is
  skepticism about the FFG's and how they relate to diet (Revisiting RCC
  sites in ID Rosi and Baxter are co-authors), the trends from our sites
  suggest that applying an FFG here may be reasonable.
  
  \section*{Fish diets}\label{fish-diets}
  \addcontentsline{toc}{section}{Fish diets}
  
  \section*{Our goal is to relate light to various biotic responses using
  a bottom-up framework, what are limitations to
  this}\label{our-goal-is-to-relate-light-to-various-biotic-responses-using-a-bottom-up-framework-what-are-limitations-to-this}
  \addcontentsline{toc}{section}{Our goal is to relate light to various
  biotic responses using a bottom-up framework, what are limitations to
  this}
  
  Complex trophic dynamics and limits on primary productivity such as
  photosaturation and nutrient limitation make it hard to capture a
  per-unit-light biotic response. Studies demonstrate clear differences
  between old-growth and second-growth light dynamics and system
  productivity, but how frequently or how large of gaps are necessary to
  get a response?
  
  \chapter*{Conclusion}\label{conclusion}
  \addcontentsline{toc}{chapter}{Conclusion}
  
  Gaps do have an effect but it appears to be local -- at least for gaps
  of this size and the effect is muted at the whole reach scale. But more
  gaps may give us a different answer since there is clearly some local
  impact.
  
  \pagebreak
  
  \chapter*{References}\label{references}
  \addcontentsline{toc}{chapter}{References}
  
  \hypertarget{refs}{}
  \hypertarget{ref-Bechtold2012}{}
  Bechtold, H. A., Rosi-Marshall, E. J., Warren, D. R., \& Cole, J. J.
  (2012). A practical method for measuring integrated solar radiation
  reaching streambeds using photodegrading dyes. \emph{Freshwater
  Science}, \emph{31}(4), 1070--1077.
  
  \hypertarget{ref-Cole2003}{}
  Cole, M. B., Russell, K. R., \& Mabee, T. J. (2003). Relation of
  headwater macroinvertebrate communities to in-stream and adjacent stand
  characteristics in managed second-growth forests of the oregon coast
  range mountains. \emph{Canadian Journal of Forest Research},
  \emph{33}(8), 1433--1443.
  
  \hypertarget{ref-Heaston2018}{}
  Heaston, E. D., Kaylor, M. J., \& Warren, D. R. (2018). Aquatic food web
  response to patchy shading along forested headwater streams.
  \emph{Canadian Journal of Fisheries and Aquatic Sciences},
  \emph{75}(12), 2211--2220.
  
  \hypertarget{ref-Jacobs1974}{}
  Jacobs, J. (1974). Quantitative measurement of food selection.
  \emph{Oecologia}, \emph{14}(4), 413--417.
  
  \hypertarget{ref-Kaylor2017Eco}{}
  Kaylor, M. J., \& Warren, D. R. (2017). Linking riparian shade and the
  legacies of forest management to fish and vertebrate biomass in forested
  streams. \emph{Ecosphere}, \emph{8}(6), e01845.
  
  \hypertarget{ref-Kaylor2017FS}{}
  Kaylor, M. J., Warren, D. R., \& Kiffney, P. M. (2017). Long-term
  effects of riparian forest harvest on light in pacific northwest (USA)
  streams. \emph{Freshwater Science}, \emph{36}(1), 1--13.
  
  \hypertarget{ref-Kreutzweiser2012}{}
  Kreutzweiser, D. P., Sibley, P. K., Richardson, J. S., \& Gordon, A. M.
  (2012). Introduction and a theoretical basis for using disturbance by
  forest management activities to sustain aquatic ecosystems.
  \emph{Freshwater Science}, \emph{31}(1), 224--231.
  
  \hypertarget{ref-Kruskal1964}{}
  Kruskal, J. B. (1964). Multidimensional scaling by optimizing goodness
  of fit to a nonmetric hypothesis. \emph{Psychometrika}, \emph{29}(1),
  1--27.
  
  \hypertarget{ref-Lau2009}{}
  Lau, D. C. P., Leung, K. M. Y., \& Dudgeon, D. (2009). What does stable
  isotope analysis reveal about trophic relationships and the relative
  importance of allochthonous and autochthonous resources in tropical
  streams? A synthetic study from hong kong. \emph{Freshwater Biology},
  \emph{54}(1), 127--141.
  
  \hypertarget{ref-liess2012}{}
  Liess, A., Le Gros, A., Wagenhoff, A., Townsend, C. R., \& Matthaei, C.
  D. (2012). Landuse intensity in stream catchments affects the benthic
  food web: Consequences for nutrient supply, periphyton c: Nutrient
  ratios, and invertebrate richness and abundance. \emph{Freshwater
  Science}, \emph{31}(3), 813--824.
  
  \hypertarget{ref-Macarelli2011}{}
  Macarelli, A., \& others. (2011). Quantity and quality: Unifying food
  web and ecosystem perspectives on the role of resource subsidies in
  freshwater. \emph{Ecology}, \emph{92}, 1215--1225.
  
  \hypertarget{ref-PC-ORD}{}
  McCune, B., \& Mefford, M. J. (2016). \emph{PC-ord. multivariate
  analysis of ecological data. version 7.} Gleneden Beach, Oregon, U.S.A.:
  MjM Software Design.
  
  \hypertarget{ref-McCune2002}{}
  McCune, B., Grace, J. B., \& Urban, D. L. (2002). \emph{Analysis of
  ecological communities} (Vol. 28). MjM software design Gleneden Beach,
  OR.
  
  \hypertarget{ref-McCutchan2002}{}
  McCutchan, J. H. J., \& Lewis, W. M. J. (2002). Relative importance of
  carbon sources for macroinvertebrates in a rocky mountain stream.
  \emph{Limnology and Oceanography}, \emph{47}(3), 742--752.
  
  \hypertarget{ref-Merritt2008}{}
  Merritt, R. W., Cummins, K. W., \& Berg, M. B. (2008). \emph{An
  introduction to the aquatic insects of north america}. Kendall Hunt.
  
  \hypertarget{ref-Murphy1981}{}
  Murphy, M. L., \& Hall, J. D. (1981). Vaired effects of clear-cut
  logging on predators and their habitat in small streams of the cascade
  mountains, oregon. \emph{Canadian Journal of Fisheries and Aquatic
  Sciences}, \emph{38}(2), 137--145.
  
  \hypertarget{ref-vegan}{}
  Oksanen, J., Blanchet, F. G., Friendly, M., Kindt, R., Legendre, P.,
  McGlinn, D., \ldots{} Wagner, H. (2018). \emph{Vegan: Community ecology
  package}. Retrieved from \url{https://CRAN.R-project.org/package=vegan}
  
  \hypertarget{ref-Pan2011}{}
  Pan, Y., Chen, J. M., Birdsey, R., McCullough, K., He, L., \& Deng, F.
  (2011). Age structure and disturbance legacy of north american forests.
  \emph{Biogeosciences. 8: 715-732.}, \emph{8}, 715--732.
  
  \hypertarget{ref-Peckarsky1998}{}
  Peckarsky, B. L., \& McIntosh, A. R. (1998). Fitness and community
  consequences of avoiding multiple predators. \emph{Oecologia},
  \emph{113}(4), 565--576.
  
  \hypertarget{ref-Purcell2009}{}
  Purcell, A. H., Bressler, D. W., Paul, M. J., Barbour, M. T., Rankin, E.
  T., Carter, J. L., \& Resh, V. H. (2009). Assessment tools for urban
  catchments: Developing biological indicators based on benthic
  macroinvertebrates 1. \emph{JAWRA Journal of the American Water
  Resources Association}, \emph{45}(2), 306--319.
  
  \hypertarget{ref-R-base}{}
  R Core Team. (2018). \emph{R: A language and environment for statistical
  computing}. Retrieved from \url{https://www.R-project.org/}
  
  \hypertarget{ref-Syrjanen2011}{}
  Syrjänen, J., Korsu, K., Louhi, P., Paavola, R., \& Muotka, T. (2011).
  Stream salmonids as opportunistic foragers: The importance of
  terrestrial invertebrates along a stream-size gradient. \emph{Canadian
  Journal of Fisheries and Aquatic Sciences}, \emph{68}(12), 2146--2156.
  
  \hypertarget{ref-Vannote1980}{}
  Vannote, R. L., Minshall, G. W., Cummins, K. W., Sedell, J. R., \&
  Cushing, C. E. (1980). The river continuum concept. \emph{Canadian
  Journal of Fisheries and Aquatic Sciences}, \emph{37}(1), 130--137.
  
  \hypertarget{ref-Warren2017}{}
  Warren, D. R., Collins, S. M., Purvis, E. M., Kaylor, M. J., \&
  Bechtold, H. A. (2017). Spatial variability in light yields colimitation
  of primary production by both light and nutrients in a forested stream
  ecosystem. \emph{Ecosystems}, \emph{20}(1), 198--210.
  
  \hypertarget{ref-Warren2016}{}
  Warren, D. R., Keeton, W. S., Kiffney, P. M., Kaylor, M. J., Bechtold,
  H. A., \& Magee, J. (2016). Changing forests---changing streams:
  Riparian forest stand development and ecosystem function in temperate
  headwaters. \emph{Ecosphere}, \emph{7}(8).
  
  \hypertarget{ref-Wilzbach1986}{}
  Wilzbach, M. A., Cummins, K. W., \& Hall, J. D. (1986). Influence of
  habitat manipulations on interactions between cutthroat trout and
  invertebrate drift. \emph{Ecology}, \emph{67}(4), 898--911.
  
  \hypertarget{ref-Wipfli1997}{}
  Wipfli, M. S. (1997). Terrestrial invertebrates as salmonid prey and
  nitrogen sources in streams: Contrasting old-growth and young-growth
  riparian forests in southeastern alaska, usa. \emph{Canadian Journal of
  Fisheries and Aquatic Sciences}, \emph{54}(6), 1259--1269.
  
  \hypertarget{ref-Wootton2012}{}
  Wootton, J. T. (2012). River food web response to large-scale riparian
  zone manipulations. \emph{PLoS One}, \emph{7}(12), e51839.


\end{document}

